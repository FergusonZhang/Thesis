\pagenumbering{gobble}
\addtocontents{toc}{\protect\setcounter{tocdepth}{0}}
\title{The Computational Detection of Short-term Balancing Selection in \emph{Capsella grandiflora}}
\subtitle{}
\names{Zhenrui}{}{Zhang}
\submitdate{Feb}{2022}
\degree{Biology}{Bio}
\advisor{Dr. Robert Williamson}
\maketitle
\abstractpreamble

\begin{abstract}

In spite of the increasing interest in the process of balancing selection, its frequency in species and evolutionary significance remain unclear due to challenges of its detection. In order to detect evidence of short-term balancing selection in \emph{Capsella grandiflora}, the whole genome data, including all eight chromosomes, of 177 individuals from a single \emph{C. grandiflora} population was analyzed. The pairwise differences was used to calculate Tajima’s D, a standard score of natural selection. The top 50 windows with highest \emph{D} values were converted to 46 gene candidates using the \emph{Capsella rubella} reference genome. Additionally, another test for balancing selection based on linkage disequilibrium, the integrated haplotype homozygosity pooled test, was applied to same data set with an assumption of uniform recombination possibility across each chromosome, and 28 of the 50 windows were further analyzed using the extended haplotype homozygosity test. Finally, the 46 gene candidates were run through a gene ontology enrichment analysis using \emph{Arabidopsis thaliana} gene orthologues. The results of these tests were consistent with the existence of short-term balancing selection in \emph{C. grandiflora}, and those gene candidates under balancing selection were found to involve in a variety of biological functions, including membrane transportation, disease resistance, immune response, and metabolism.
\end{abstract}

\newpage
\section{Acknowledgements}
This thesis would not have been possible, nor would my experience as an undergraduate student been as pleasant, without a number of people. My thesis advisor, Dr. Robert Williamson, has been a wonderful mentor and friend. I am very grateful for his generosity, encouragement, respect, thoughtful guidance, and patience when I made silly mistakes. I also want to thank Dr. Emily Josephs at the Michigan State University for generously sharing the \emph{Capsella grandiflora} genome data. This thesis also benefited greatly from the advice of other BBE faculties, as well as my dear biology major fellows. Finally, I want to thank Dr. Ella Ingram for holding great accountability groups, in which I learned not only how to make plans for scientific study, but also how to professionally communicate with other people.

\newpage
\addtocontents{toc}{\protect\setcounter{tocdepth}{2}}
\pagenumbering{roman}
\setcounter{page}{0}
\tableofcontents

\newpage
\cleardoublepage
\addcontentsline{toc}{section}{\listfigurename}
\listoffigures

%\newpage
%\cleardoublepage
%\addcontentsline{toc}{section}{\listtablename}
%\listoftables

\newpage
\section{List of Abbreviations}
\noindent
\begin{itemize}[label={},leftmargin=0in]
    \item CI \textemdash Confidence interval
    \item EHH \textemdash The extended haplotype homozygosity test
    \item GO \textemdash Gene ontology
    \item iHH12 \textemdash The integrated haplotype homozygosity pooled test
    \item LD \textemdash Linkage disequilibrium
    \item SNP \textemdash Single nucleotide polymorphism 
    \item SI \textemdash Self-incompatibility
    \item VCF \textemdash Variant call format file
\end{itemize}

\newpage
\section{List of Symbols}
\noindent
\textbf{English Symbols}\\
 \emph{D}\tabto{2cm}The Tajima's D\\
 \emph{k}\tabto{2cm}Pairwise differences (nucleotide diversity)\\
 p\tabto{2cm}The p-value of significance test.\\

%\newpage
%\section{Glossary}
%\setlength{\parindent}{0pt}
%\textbf{Amphere} – the unit of electrical current. Current is defined as the %amount of charge that flows past a give point, per unit of time.\\

\setlength{\parindent}{15pt}